



\chapter{\#\#\#使用示例}

\section{引用}
 引用\cite{BQC_2020}引用\cite{向守平2008天体物理概论}引用\cite{2019First},引用\cite{BQC_2020,向守平2008天体物理概论,2019First}


\section{图片}

% 复制以下代码
\begin{figure}[htbp]
	\centering
	\includegraphics[width=0.5\linewidth]{example-image-a}%
	\caption{这是一个示例图片}
	\label{fig:一个示例图片}
\end{figure}

% \zhlipsum




% 需要在导入这两个包
%'''
% \usepackage{subcaption}
% \usepackage{graphicx}
%'''


\begin{figure}
    \centering
    \begin{subfigure}{0.49\textwidth}
        \centering
        \includegraphics[width=\linewidth]{example-image-a}
        \caption{fig:一个示例图片1}
        \label{label1}
    \end{subfigure}
    \begin{subfigure}{0.49\textwidth}
        \centering
        \includegraphics[width=\linewidth]{example-image-a}
        \caption{fig:一个示例图片2}
        \label{label2}
    \end{subfigure}
    \caption{两张图片}
\end{figure}



\section{表格}
\begin{table}[htbp]
	\centering
	\caption{An example Table.}
	\label{tab:example-tab}
	\begin{tabular}{lcc}
		\toprule
		Star           & Mass        & Luminosity  \\
		               & $M_{\odot}$ & $L_{\odot}$ \\
		\midrule
		Sun            & 1.00        & 1.00        \\
		$\alpha$~Cen~A & 1.10        & 1.52        \\
		$\epsilon$~Eri & 0.82        & 0.34        \\
		\bottomrule
	\end{tabular}
\end{table}



\zhlipsum
